\documentclass[letterpaper,11pt]{article}

\usepackage{latexsym}
\usepackage[empty]{fullpage}
\usepackage{titlesec}
\usepackage{marvosym}
\usepackage[usenames,dvipsnames]{color}
\usepackage{verbatim}
\usepackage{enumitem}
\usepackage[pdftex]{hyperref}
\usepackage{fancyhdr}

\usepackage{ragged2e}

\pagestyle{fancy}
\fancyhf{} % clear all header and footer fields
\fancyfoot{}
\renewcommand{\headrulewidth}{0pt}
\renewcommand{\footrulewidth}{0pt}

% Adjust margins
\addtolength{\oddsidemargin}{-0.375in}
\addtolength{\evensidemargin}{-0.375in}
\addtolength{\textwidth}{1in}
\addtolength{\topmargin}{-.5in}
\addtolength{\textheight}{1.0in}

\urlstyle{same}

\raggedbottom
\raggedright
\setlength{\tabcolsep}{0in}

% Sections formatting
%\titleformat{\section}{
	%  \vspace{-4pt}\scshape\raggedright\large
	%}{}{0em}{}[\color{black}\titlerule \vspace{-5pt}]
%\titleformat{\section}{
	%	\vspace{-4pt}\scshape\bfseries\raggedright\large
	%}{}{0em}{}[\color{black}\titlerule \vspace{-5pt}]
\titleformat{\section}{
	\vspace{-4pt}\bfseries\MakeUppercase\raggedright\Large
}{}{0em}{}[\color{black}\titlerule \vspace{-5pt}]

%-------------------------
% Custom commands
\newcommand{\resumeItem}[2]{
	\item\small{
		\textbf{#1}{: #2 \vspace{-2pt}}
	}
}
\newcommand{\resumeItemAlt}[2]{
	\vspace{-2pt}
	\item\small{
		\textbf{#1}{#2 \vspace{-2pt}}
	}
}


\newcommand{\resumeSubheading}[4]{
	\vspace {-1pt}\item
	\begin{tabular*}{0.97\textwidth}{l@{\extracolsep{\fill}}r}
		\textbf{#1} & #2\\
		\textit{\small#3} & \textit{\small #4} \\
	\end{tabular*}\vspace{-5pt}
}

\newcommand{\resumeSubheadingAlt}[6]{
	\vspace{-1pt}\item
	\begin{tabular*}{0.97\textwidth}{l@{\extracolsep{\fill}}r}
		\textbf{#1} & #2\\
		\textit{\small#3} & \textit{\small #4} \\
		\textit{\small#5} & \textit{\small #6} \\
	\end{tabular*}\vspace{-5pt}
}

\newcommand{\resumeProject}[2]{
	\vspace {-1pt}\item
	\begin{tabular*}{0.97\textwidth}{l@{\extracolsep{\fill}}r}
		\textbf{#1}\\
		{\small\href{#2}{#2}} \\
	\end{tabular*}\vspace{-5pt}
}

\newcommand{\resumeSubItem}[2]{\resumeItem{#1}{#2}\vspace{-4pt}}

\renewcommand{\labelitemii}{$\circ$}

\newcommand{\resumeSubHeadingListStart}{\begin{itemize}[leftmargin=*]}
	\newcommand{\resumeSubHeadingListEnd}{\end{itemize}}
\newcommand{\resumeItemListStart}{\begin{itemize} }
	\newcommand{\resumeItemListEnd}{\end{itemize}\vspace{-5pt}}


\usepackage{setspace}

\renewcommand{\baselinestretch}{1.5} 


%-------------------------------------------
%%%%%%  CV STARTS HERE  %%%%%%%%%%%%%%%%%%%%%%%%%%%%

%-------------------------------------------
%%%%%%  CV INICIA AQUI  %%%%%%%%%%%%%%%%%%%%%%%%%%%%

\begin{document}
	
	%\justify
	
	\setstretch{1.5}
	
	\begin{tabular*}{\textwidth}{l@{\extracolsep{\fill}}r}
		\textbf{\LARGE Yuri Pereira Marca} & Jaraguá do Sul - SC, Brasil
		\\
		& Telefone: +55 48 99976-2173 \\
		& E-mail: \href{mailto:yurimarca@gmail.com}{yurimarca@gmail.com} \\
		& GitHub: \href{https://github.com/yurimarca}{https://github.com/yurimarca}\\
		& LinkedIn: \href{https://www.linkedin.com/in/yurimarca/}{https://www.linkedin.com/in/yurimarca/}
		
	\end{tabular*}
	
	\section{Carta de Apresentação}
	
	Sou um Cientista de Dados com forte experiência em machine learning, otimização e MLOps, focado no desenvolvimento de soluções escaláveis de IA e na implantação de modelos em produção. Sou formado em Engenharia Eletrônica e mestre em Sistemas de Informação, com experiência acadêmica internacional no Japão, Reino Unido e Canadá. Essa trajetória diversificada me proporcionou habilidades analíticas avançadas, permitindo compreender e implementar rapidamente pesquisas de ponta, aplicando métodos científicos para solucionar desafios complexos da indústria.
	
	Na minha experiência profissional, trabalhei extensivamente com frameworks de machine learning e deep learning para treinar e ajustar modelos voltados para modelagem preditiva, detecção de anomalias e visão computacional. Minha expertise em MLOps me permite projetar e implementar pipelines completos, garantindo reprodutibilidade e escalabilidade. Utilizando ferramentas como MLflow, Docker e plataformas em nuvem como AWS e DigitalOcean, implantei modelos de machine learning com sucesso e os integrei a sistemas de produção via APIs.
	
	Com uma base sólida em análise estatística e otimização, tenho um interesse especial em IA Generativa e LLMs, explorando ativamente seu potencial em aplicações reais. Estou motivado a contribuir para soluções inovadoras de IA e ansioso para colaborar com profissionais e organizações que impulsionam a inovação em ambientes de produção.
	
	%-----------EXPERIÊNCIA PROFISSIONAL-----------------
	\section{Experiência Profissional}
	\resumeSubHeadingListStart
	
	\resumeSubheading
	{WEG}{Jaraguá do Sul, Brasil}
	{Cientista de Dados}{Jun. 2024 -- Presente}
	\resumeItemListStart
	\resumeItem{Manutenção Preditiva \& Detecção de Anomalias}
	{Desenvolvimento e implementação de modelos de machine learning e deep learning para manutenção preditiva e monitoramento de condição de turbinas eólicas.}
	\resumeItem{MLOps Completo}
	{Desenvolvimento de um pipeline escalável e automatizado de MLOps utilizando MLflow, Docker e AWS, garantindo reprodutibilidade no treinamento, implantação e monitoramento contínuo de modelos em produção.}
	\resumeItem{Implantação de IA \& Integração com Nuvem}
	{Implantação de modelos de ML em produção via APIs, otimizando a inferência em tempo real para detecção de anomalias e análise preditiva.}
	\resumeItemListEnd
	
	\resumeSubheading
	{Macnica DHW}{Florianópolis, Brasil}
	{Cientista de Dados}{Nov. 2022 -- Out. 2023}
	\resumeItemListStart
	\resumeItem{Aprendizado Supervisionado}
	{Desenvolvimento de modelos de machine learning para classificação de dados e suporte ao desenvolvimento de sensores inteligentes.}
	\resumeItem{APIs de Visão Computacional}
	{Criação e implantação de APIs para aplicações de visão computacional, incluindo uma API de anonimização que remove pessoas de vídeos.}
	\resumeItem{Desenvolvimento de IoT}
	{Liderança no desenvolvimento de um produto IoT, incluindo a implementação do protocolo MQTT e o armazenamento de dados temporais no InfluxDB.}
	\resumeItemListEnd
	
	\resumeSubheading
	{Universidade de Warwick, Warwick Business School}{Coventry, Reino Unido}
	{Pesquisa de Doutorado}{Out. 2019 -- Abr. 2022}
	\resumeItemListStart
	\resumeItem{Área de Pesquisa}
	{Ranking \& Seleção; Otimização Bayesiana; Aprendizado por Reforço.}
	\resumeItem{Pesquisa}
	{Sob supervisão de Juergen Branke, trabalhei no aprendizado eficiente de políticas em problemas de decisão sequenciais, aprimorando a eficiência amostral do Monte Carlo Tree Search (MCTS).}
	\resumeItem{Colaboração}
	{Trabalho conjunto com Prof. Chun-Hung Chen (George Mason University, EUA) em abordagens Bayesiana para tomada de decisão ótima.}
	\resumeItemListEnd

\resumeSubheading
{Shinshu University}{Nagano, Japão}
{Pesquisa de Mestrado}{Abr. 2017 -- Mar. 2019}
\resumeItemListStart
\resumeItem{Área de Pesquisa}
{Otimização Multiobjetivo; Algoritmos Genéticos; Computação Evolutiva.}
\resumeItem{Pesquisa}
{Estudo do impacto das topologias do conjunto de Pareto no desempenho de algoritmos evolutivos multiobjetivo (MOEAs) e desenvolvimento de um novo método para melhorar sua eficiência.}
\resumeItem{Colaboração Internacional}
{Trabalho com pesquisadores do Japão, México e França.}
\resumeItem{Reconhecimento}
{Recebeu o prêmio \textbf{Student Best Paper Award} na conferência EMO-2019 (Michigan State University, EUA).}
\resumeItemListEnd

\resumeSubheadingAlt
{SSE Gridtech}{Curitiba, Brasil}
{Estágio}{Out. 2014 -- Dez. 2015}       
{Engenheiro de P\&D}{Jan. 2016 -- Ago. 2016}
\resumeItemListStart
\resumeItem{Soluções para Smart Grid e IoT}
{Desenvolvimento de um sistema de comunicação LoRaWAN de longo alcance para otimizar a infraestrutura de medição inteligente, reduzindo a dependência de GSM.}
\resumeItemListEnd

  \resumeSubHeadingListEnd

%-----------EDUCAÇÃO-----------------
\section{Educação}
\resumeSubHeadingListStart
\resumeSubheading
{Shinshu University}{Nagano, Japão}
{Mestre em Engenharia Eletrônica e Sistemas de Informação}{Abr. 2017 -- Mar. 2019}
\resumeItemListStart
\resumeItemAlt{}
{Bolsa \textbf{MEXT Monbukagakusho} (Governo Japonês)}
\resumeItemListEnd
\resumeSubheading
{Concordia University}{Montreal, Canadá}
{Intercâmbio, Engenharia Elétrica}{Ago. 2012 -- Ago. 2013}
\resumeItemListStart
\resumeItemAlt{}
{Bolsa \textbf{Ciência Sem Fronteiras} (Governo Brasileiro)}
\resumeItemListEnd
\resumeSubheading
{Universidade Tecnológica Federal do Paraná}{Curitiba, Brasil}
{Bacharelado em Engenharia Eletrônica}{Jan. 2010 -- Jul. 2016}
\resumeSubHeadingListEnd
  
	%--------EXPERTISE TÉCNICA------------
	\section{Expertise Técnica}
	
	\begin{tabular}{l @{\quad} l}
		\textbf{Linguagens de Programação:}& Python, C, C++, Bash (Unix Shell).\\
		\textbf{Frameworks de Machine Learning:} & PyTorch, scikit-learn, XGBoost, FastAI.\\
		\textbf{MLOps \& Implantação:}& Docker, MLflow, Hydra, Git, CI/CD (GitLab, GitHub Actions), FastAPI.\\
		\textbf{Engenharia de Dados:}& Dask, PostgreSQL, Redshift, InfluxDB, SQL.\\
		\textbf{Plataformas em Nuvem:}& AWS (SageMaker, ECR, S3, EC2), DigitalOcean.\\
		\textbf{Otimização:}& Otimização Bayesiana, Algoritmos Evolutivos Multiobjetivo. \\
		\textbf{Sistemas Operacionais:}& Linux (Fedora, Ubuntu), Windows.\\
	\end{tabular}
	
	%--------IDIOMAS------------
	\section{Idiomas}
	
	\begin{tabular}{l @{\quad} l}
		\textbf{Português:}&Nativo\\
		\textbf{Inglês:}&Fluente \\
		\textbf{Japonês:}&Conversação \\
	\end{tabular}

%-----------PROJETOS-----------------
\section{Projetos}

\resumeSubHeadingListStart
\resumeProject
{Pipeline de Machine Learning para Predição de Preços de Aluguéis de Curto Prazo}{https://github.com/yurimarca/build-ml-pipeline-for-short-term-rental-prices}
\begin{itemize}
	\item Desenvolvimento de um pipeline de Machine Learning reprodutível para prever preços de aluguéis de curto prazo em Nova York, garantindo escalabilidade para atualizações semanais de dados.
	\item Foram Implementados processos de ingestão, limpeza, validação e engenharia de atributos para preparar os dados para modelagem.
	\item Treinamento e otimização de modelos de regressão Random Forest, aplicando ajuste de hiperparâmetros para melhorar a performance preditiva.
	\item \textbf{MLflow} foi utilizado para rastreamento de experimentos e gerenciamento de modelos, além de integrar \textbf{Weights \& Biases} para rastreamento e visualização de artefatos.
	\item Projeto inclui um pipeline para re-treinamento contínuo com novos dados, facilitando a melhoria constante do modelo e sua implantação.
\end{itemize}

\resumeSubHeadingListEnd

%-----------PUBLICATIONS-----------------
\section{Publicações}
\resumeItemListStart
\resumeItemAlt{}
{\textbf{Y. Marca}, H. Aguirre, S. Zapotecas, A. Liefooghe, B. Derbel, S. Verel, and K. Tanaka. Approximating Pareto set topology by cubic interpolation on bi-objective problems. 10th International Conference on Evolutionary Multi-Criterion Optimization (EMO 2019), Lecture Notes in Computer Science (LNCS), Michigan, USA. \textbf{(Best Student Paper Award)}}	
\resumeItemAlt{}
{\textbf{Y. Marca}, H. Aguirre, S. Zapotecas, A. Liefooghe, B. Derbel, S. Verel, and K. Tanaka. NSGA-II with Spline Interpolation on Bi-objective Problems with Difficult Pareto Set Topology. JPNSEC 2018 Symposium on Evolutionary Computation, Fukuoka, 2018. \textbf{(Young Research Award)}}
\resumeItemAlt{}
{\textbf{Y. Marca}, H. Aguirre, S. Zapotecas, A. Liefooghe, B. Derbel, S. Verel, and K. Tanaka. MOEAs on Problems with Difficult Pareto Set Topologies. IEICE Shin-etsu Branch IEEE Session, Niigata University, 2018, p. 169.\textbf{(Young Research Award)}}
\resumeItemAlt{}
{\textbf{Y. Marca}, H. Aguirre, S. Zapotecas, A. Liefooghe, B. Derbel, S. Verel, and K. Tanaka. Pareto dominance-based MOEAs on problems with difficult pareto set topologies. In Proceedings of the Genetic and Evolutionary Computation Conference Companion (GECCO '18). ACM, New York, NY, USA, 189-190.}
\resumeItemAlt{}
{C. E. A. L. Rocha, \textbf{Y. P. Marca}, F. K. Schneider. Support Platform for Decision-Making in Research and Technological Development in Public Health. ESPACIOS (CARACAS), v. 39, p. 14-26, 2018.}

\resumeItemAlt{}
{\textbf{Y. P. Marca}, S. Scholze. Proposta de Substituição da Comunicação GSM em Smart Grids por Rádios de Longo Alcance. XXXIII Simpósio Brasileiro de Telecomunicações, 2015, Juiz de Fora, MG. Anais Completo da Programação Técnica, 2015.}

\resumeItemAlt{}
{\textbf{Y. P. Marca}, C. E. A. L. Rocha, B. Schneider Jr , F. K. Schneider. Plataforma de Apoio ao Processo Decisório em Pesquisa e Desenvolvimento Tecnológico em Saúde. Congresso Brasileiro de Engenharia Biomédica, 2012, Porto de Galinhas. ANAIS - CBEB 2012, 2012.}
\resumeItemAlt{}
{M. P. Krause, D. M. Nakato, \textbf{Y. P. Marca}, F. K. Schneider. Gerenciamento do Controle da Glicemia Utilizando um Aplicativo para Celular. Congresso Brasileiro de Engenharia Biomédica, 2012, Porto de Galinhas. ANAIS - CBEB 2012, 2012.}

\resumeItemListEnd

\end{document}
